% Intended LaTeX compiler: pdflatex
\documentclass[11pt]{article}
\usepackage[utf8]{inputenc}
\usepackage[T1]{fontenc}
\usepackage{graphicx}
\usepackage{grffile}
\usepackage{longtable}
\usepackage{wrapfig}
\usepackage{rotating}
\usepackage[normalem]{ulem}
\usepackage{amsmath}
\usepackage{textcomp}
\usepackage{amssymb}
\usepackage{capt-of}
\usepackage{hyperref}
\author{Jay Dixit}
\date{\today}
\title{}
\hypersetup{
 pdfauthor={Jay Dixit},
 pdftitle={},
 pdfkeywords={},
 pdfsubject={},
 pdfcreator={Emacs 25.1.1 (Org mode 9.1.13)}, 
 pdflang={English}}
\begin{document}

\tableofcontents
\newpage
(provide 'modest-ruin)

(add-to-list 'org-latex-classes
  '(``modest-ruin''
``

$\backslash$\documentclass[12pt]{article}
\% $\backslash$\usepackage[includeheadfoot,margin=1.0in,hmargin=1.0in,vmargin=0.5in]{geometry} \% for normal margins
$\backslash$\usepackage[includeheadfoot,margin=1.5in,hmargin=1.5in,vmargin=0.5in]{geometry} \% for insanely wide margins
\% $\backslash$\usepackage[includeheadfoot,margin=2.0in,hmargin=2.0in,vmargin=0.5in]{geometry} \% for insanely wide margins
$\backslash$\usepackage{float}

$\backslash$\usepackage{algorithm}
$\backslash$\usepackage{amsmath}
$\backslash$\usepackage{ifxetex}
$\backslash$\ifxetex
  $\backslash$\usepackage{fontspec,xltxtra,xunicode}
  $\backslash$\defaultfontfeatures{Mapping=tex-text,Scale=MatchLowercase}
  $\backslash$\setromanfont{Garamond Premier Pro}
\%  $\backslash$\setromanfont{Adobe Caslon Pro}
 $\backslash$\setsansfont{Helvetica Neue}
  $\backslash$\setmonofont{Myriad Pro}
$\backslash$\else
  $\backslash$\usepackage[mathletters]{ucs}
  $\backslash$\usepackage[utf8x]{inputenc}
$\backslash$\fi
$\backslash$\usepackage{url}
$\backslash$\usepackage{paralist}
$\backslash$\usepackage{graphicx}
$\backslash$\usepackage{tikz}
$\backslash$\usepackage{calc}
$\backslash$\usepackage{eso-pic}
$\backslash$\usepackage{etoolbox}
$\backslash$\usepackage{xcolor}
$\backslash$\PassOptionsToPackage{hyperref,x11names}{xcolor}
$\backslash$\definecolor{blackedyblack}{HTML}{000000}
$\backslash$\definecolor{pinterestred}{HTML}{C92228}
$\backslash$\definecolor{ulyssesbutterflyblue}{HTML}{1464F4}
$\backslash$\definecolor{signalflare}{HTML}{FB782C}
$\backslash$\definecolor{niceorange}{HTML}{77CC6D}
$\backslash$\definecolor{highlighteryellow}{HTML}{FFFF01}
$\backslash$\definecolor{ghostlygrey}{HTML}{000000}
$\backslash$\definecolor{firstcolor}{HTML}{00ADEF}
$\backslash$\definecolor{secondcolor}{HTML}{DD3E74}
$\backslash$\definecolor{periodblue}{HTML}{12239e}
$\backslash$\definecolor{denimblue}{HTML}{3A5F90}
$\backslash$\definecolor{electricblue}{HTML}{05ADF3}


$\backslash$\newtoks$\backslash$\leftheader 
$\backslash$\newtoks$\backslash$\leftheaderurl
$\backslash$\newtoks$\backslash$\coverimage


$\backslash$\hyphenpenalty=5000 
$\backslash$\tolerance=1000

\%This macro is to make cleaner the specification of the titling font
$\backslash$\newfontfamily$\backslash$\mytitlefont[Color=\{highlighteryellow\}]\{Gotham Narrow Bold\}
$\backslash$\newfontfamily$\backslash$\myauthorfont[Color=\{highlighteryellow\}]\{Gotham Narrow Bold\}
$\backslash$\newfontfamily$\backslash$\mybluefont[Color=electricblue]{Gotham Narrow Bold}
$\backslash$\DeclareTextFontCommand{\\textbf}\{$\backslash$\bfseries$\backslash$\color{electricblue}\}
$\backslash$\DeclareTextFontCommand{\\textit}{\\itshape}


$\backslash$\usepackage{textcase}

$\backslash$\pagenumbering{arabic}
$\backslash$\makeatletter

\%This macro now controls the position of the background pic
\%Please do not change from here
$\backslash$\newcommand$\backslash$\BackgroundPic\{\%
$\backslash$\put(0,0)\{\%
$\backslash$\parbox[b][\\paperheight]{\\paperwidth}\{\%
$\backslash$\vfill
$\backslash$\centering
\%inside the tikzpicture environment, you can do anything you want with the image
$\backslash$\begin{tikzpicture}

$\backslash$\node [inner sep=0pt,outer sep=0pt] at (0,0) \{$\backslash$\includegraphics[width=\\paperwidth,height=\\paperheight]{\\the\\coverimage}\};

$\backslash$\node at  (0,5) [opacity=1.0] \{$\backslash$\parbox[b][0.5\\textheight]{\\textwidth}\{\%
  $\backslash$\begin{raggedright}
  $\backslash$\leavevmode
    $\backslash$\vskip 1cm
  \{$\backslash$\mytitlefont$\backslash$\fontsize{75}{85}$\backslash$\bfseries{\\@title}$\backslash$\par\}
    $\backslash$\vskip 1cm

\%\{$\backslash$\myauthorfont$\backslash$\fontsize{30}{40}\{\{$\backslash$\bfseries{\\@degree}$\backslash$\par\}\}\}

$\backslash$\vfill
$\backslash$\end{raggedright}\}\};
$\backslash$\node at (0,-8) [opacity=1] \{$\backslash$\parbox[b][0.3\\textheight]{\\textwidth}\{\%
$\backslash$\begin{raggedright}
$\backslash$\vfill
\{$\backslash$\myauthorfont$\backslash$\Large \{$\backslash$\@author\}\}
    $\backslash$\newline
          \{$\backslash$\myauthorfont$\backslash$\Large $\backslash$\href{mailto:jay@storytelling.nyc}{jay@storytelling.nyc}\}
        $\backslash$\newline
\{$\backslash$\myauthorfont$\backslash$\Large
$\backslash$\href{http://storytelling.nyc}{Storytelling.NYC}\}
$\backslash$\newline
\{$\backslash$\myauthorfont$\backslash$\Large © 2018 $\backslash$\@author\}
    $\backslash$\newline
\%\{$\backslash$\myauthorfont$\backslash$\Large Private and Confidential\}
 \%   $\backslash$\newline
        $\backslash$\newline
    \{$\backslash$\myauthorfont$\backslash$\Large $\backslash$\@date$\backslash$\par\}
\%\{$\backslash$\myauthorfont$\backslash$\Large $\backslash$\href{http://jaydixit.com}{\\@degree }\}
$\backslash$\end{raggedright}
\}\};
$\backslash$\end{tikzpicture}
\%Don't change
$\backslash$\vfill
\}\}\}
\%This macro executes a hook at the beginning of the document that  puts the background correctly. 
$\backslash$\AtBeginDocument\{$\backslash$\AddToShipoutPicture*{\\BackgroundPic}\}
$\backslash$\AtBeginDocument\{$\backslash$\globalcolor{ghostlygrey}\}



\%The maketitle macro now only includes the titling and not the background. 
$\backslash$\def$\backslash$\maketitle\{ $\backslash$\newgeometry{margin=1in} $\backslash$\thispagestyle{empty} $\backslash$\vfill $\backslash$\null $\backslash$\cleardoublepage$\backslash$\restoregeometry\}



$\backslash$\setcounter{secnumdepth}{0}




$\backslash$\usepackage{fancyhdr}
$\backslash$\pagestyle{fancy}
$\backslash$\renewcommand{\\sectionmark}[1]\{$\backslash$\markboth{#1}{}\}
$\backslash$\lhead\{$\backslash$\href{\\the\\leftheaderurl}{\\the\\leftheader}\}
$\backslash$\chead{}
$\backslash$\rhead\{\{$\backslash$\nouppercase{\\leftmark}\}\}
\% $\backslash$\rhead\{$\backslash$\@title: \{$\backslash$\nouppercase{\\leftmark}\}\}
$\backslash$\lfoot{}
$\backslash$\cfoot{}
$\backslash$\rfoot{}
$\backslash$\usepackage{listings}
$\backslash$\setlength{\\parindent}{0pt}
$\backslash$\setlength{\\parskip}{12pt plus 2pt minus 1pt} \% space between paragraphs

\% spacing: how to read \{12pt plus 4pt minus 2pt\}
\%           12pt is what we would like the spacing to be
\%           plus 4pt means that \TeX{} can stretch it by at most 4pt
\%           minus 2pt means that \TeX{} can shrink it by at most 2pt
\%       This is one example of the concept of, 'glue', in \TeX{}

$\backslash$\usepackage{fancyvrb}
$\backslash$\usepackage{enumerate}
$\backslash$\usepackage{ctable}
$\backslash$\setlength{\\paperwidth}{8.5in}
$\backslash$\setlength{\\paperheight}{11in}
  $\backslash$\tolerance=1000
$\backslash$\usepackage{tocloft}
$\backslash$\renewcommand{\\cftsecleader}\{$\backslash$\cftdotfill{\\cftdotsep}\}
$\backslash$\usepackage[normalem]{ulem}


$\backslash$\makeatletter
$\backslash$\newcommand{\\globalcolor}[1]\{\%
  $\backslash$\color{#1}$\backslash$\global$\backslash$\let$\backslash$\default@color$\backslash$\current@color
\}
$\backslash$\makeatother

$\backslash$\newcommand{\\textsubscr}[1]\{$\backslash$\ensuremath\{\(_{\text{$\backslash$}\scriptsize\text{$\backslash$}\textrm{#1}}\)\}\}

$\backslash$\usepackage{enumitem}

$\backslash$\newlist{mylist}{enumerate}{10} 


\% control line spacing in bulleted list
$\backslash$\setlist\{noitemsep, topsep=-8pt, after=$\backslash$\vspace{12pt}\} \% for no spacing between list items
\% see: \url{https://tex.stackexchange.com/questions/199118/modifying-whitespace-before-and-after-list-separately-using-enumitem-package}
\%$\backslash$\setlist{topsep=0pt} \% for a line between list items


$\backslash$\renewcommand{\\labelitemi}\{$\backslash$\raise 0.25ex$\backslash$\hbox{\\tiny$\\bullet$}\}
$\backslash$\renewcommand{\\labelitemii}\{$\backslash$\raise 0.25ex$\backslash$\hbox{\\tiny$\\bullet$}\}
$\backslash$\renewcommand{\\labelitemiii}\{$\backslash$\raise 0.25ex$\backslash$\hbox{\\tiny$\\bullet$}\}
$\backslash$\renewcommand{\\labelitemiv}\{$\backslash$\raise 0.25ex$\backslash$\hbox{\\tiny$\\bullet$}\}
$\backslash$\renewcommand{\\labelitemv}\{$\backslash$\raise 0.25ex$\backslash$\hbox{\\tiny$\\bullet$}\}
$\backslash$\renewcommand{\\labelitemvi}\{$\backslash$\raise 0.25ex$\backslash$\hbox{\\tiny$\\bullet$}\}
$\backslash$\renewcommand{\\labelitemvii}\{$\backslash$\raise 0.25ex$\backslash$\hbox{\\tiny$\\bullet$}\}
$\backslash$\renewcommand{\\labelitemviii}\{$\backslash$\raise 0.25ex$\backslash$\hbox{\\tiny$\\bullet$}\}
$\backslash$\renewcommand{\\labelitemix}\{$\backslash$\raise 0.25ex$\backslash$\hbox{\\tiny$\\bullet$}\}
$\backslash$\renewcommand{\\labelitemx}\{$\backslash$\raise 0.25ex$\backslash$\hbox{\\tiny$\\bullet$}\}

$\backslash$\setlistdepth{10}
$\backslash$\setlist[itemize,1]{label=\\raise 0.25ex\\hbox\\tiny$\\bullet$}
$\backslash$\setlist[itemize,2]{label=\\raise 0.25ex\\hbox\\tiny$\\bullet$}
$\backslash$\setlist[itemize,3]{label=\\raise 0.25ex\\hbox\\tiny$\\bullet$}
$\backslash$\setlist[itemize,4]{label=\\raise 0.25ex\\hbox\\tiny$\\bullet$}
$\backslash$\setlist[itemize,5]{label=\\raise 0.25ex\\hbox\\tiny$\\bullet$}
$\backslash$\setlist[itemize,6]{label=\\raise 0.25ex\\hbox\\tiny$\\bullet$}
$\backslash$\setlist[itemize,7]{label=\\raise 0.25ex\\hbox\\tiny$\\bullet$}
$\backslash$\setlist[itemize,8]{label=\\raise 0.25ex\\hbox\\tiny$\\bullet$}
$\backslash$\setlist[itemize,9]{label=\\raise 0.25ex\\hbox\\tiny$\\bullet$}
$\backslash$\setlist[itemize,10]{label=\\raise 0.25ex\\hbox\\tiny$\\bullet$}
$\backslash$\renewlist{itemize}{itemize}{10}





$\backslash$\definecolor{azure}{HTML}{f2feff}

$\backslash$\usepackage{lipsum}
$\backslash$\usepackage{tikz}
$\backslash$\usetikzlibrary{backgrounds}
$\backslash$\makeatletter

$\backslash$\tikzset\{\%
  fancy quotes/.style=\{
    text width=$\backslash$\fq@width pt,
    align=justify,
    inner sep=1em,
    anchor=north west,
    minimum width=$\backslash$\linewidth,
  \},
  fancy quotes width/.initial=\{.8$\backslash$\linewidth\},
  fancy quotes marks/.style=\{
    scale=8,
    text=black,
    inner sep=0pt,
  \},
  fancy quotes opening/.style=\{
    fancy quotes marks,
  \},
  fancy quotes closing/.style=\{
    fancy quotes marks,
  \},
  fancy quotes background/.style=\{
    show background rectangle,
    inner frame xsep=0pt,
    background rectangle/.style=\{
      fill=azure,
      rounded corners,
    \},
  \}
\}

$\backslash$\newenvironment{fancyquotes}[1][]\{\%
$\backslash$\noindent
$\backslash$\tikzpicture[fancy quotes background]
$\backslash$\node[fancy quotes opening,anchor=north west] (fq@ul) at (0,0) \{``\};
$\backslash$\tikz@scan@one@point$\backslash$\pgfutil@firstofone(fq@ul.east)
$\backslash$\pgfmathsetmacro{\\fq@width}{\\linewidth - 2*\\pgf@x}
$\backslash$\node[fancy quotes,#1] (fq@txt) at (fq@ul.north west) $\backslash$\bgroup\}
\{$\backslash$\egroup;
$\backslash$\node[overlay,fancy quotes closing,anchor=east] at (fq@txt.south east) \{''\};
$\backslash$\endtikzpicture\}
$\backslash$\makeatother


$\backslash$\usepackage{setspace}
$\backslash$\usepackage{lipsum}
$\backslash$\usepackage{etoolbox}
$\backslash$\AtBeginEnvironment{quote}\{$\backslash$\singlespace$\backslash$\vspace{-\\topsep}$\backslash$\small\}
$\backslash$\AtEndEnvironment{quote}\{$\backslash$\vspace{-\\topsep}$\backslash$\endsinglespace\}


$\backslash$\usepackage[sc]{titlesec}
\% $\backslash$\titlespacing{command}{left spacing}{before spacing}{after spacing}[right]
$\backslash$\titlespacing*{\\section}{0pt}{6pt}{-6pt}
$\backslash$\titlespacing*{\\subsection}{0pt}{0pt}{-6pt}
$\backslash$\titlespacing*{\\subsubsection}{0pt}{6pt}{-6pt}

$\backslash$\titleformat*{\\section}\{$\backslash$\sffamily$\backslash$\fontsize{14}{14}$\backslash$\raggedright$\backslash$\bfseries$\backslash$\sffamily$\backslash$\color{blackedyblack}\}
$\backslash$\titleformat*{\\subsection}\{$\backslash$\sffamily$\backslash$\fontsize{12}{12}$\backslash$\scshape$\backslash$\color{blackedyblack}\}
$\backslash$\titleformat*{\\subsubsection}\{$\backslash$\sffamily$\backslash$\fontsize{12}{12}$\backslash$\raggedright$\backslash$\bfseries$\backslash$\color{blackedyblack}\}
$\backslash$\titleformat*{\\paragraph}\{$\backslash$\sffamily$\backslash$\sanssize$\backslash$\raggedright$\backslash$\bfseries$\backslash$\rmfamily$\backslash$\color{blackedyblack}\}
$\backslash$\titleformat*{\\subparagraph}\{$\backslash$\sffamily$\backslash$\fontsize{10}{10}$\backslash$\raggedright$\backslash$\bfseries$\backslash$\ttfamily$\backslash$\color{blackedyblack}\}
$\backslash$\usepackage[breaklinks=true,linktocpage,xetex]{hyperref} 
$\backslash$\hypersetup{colorlinks, citecolor=electricblue,filecolor=electricblue,linkcolor=electricblue,urlcolor=electricblue}




 [NO-DEFAULT-PACKAGES]
 [NO-PACKAGES]"
("$\backslash$\section{%s}`` . ''$\backslash$\section*{%s}``)
(''$\backslash$\subsection{%s}`` . ''$\backslash$\subsection*{%s}``)
(''$\backslash$\subsubsection{%s}`` . ''$\backslash$\subsubsection*{%s}``)
(''$\backslash$\paragraph{%s}`` . ''$\backslash$\paragraph*{%s}``)
(''$\backslash$\subparagraph{%s}`` . ''$\backslash$\subparagraph*{%s}``)))


(setq org-latex-to-pdf-process 
  '(``xelatex -interaction nonstopmode \%f''
     ``xelatex -interaction nonstopmode \%f'')) ;; for multiple passes
\end{document}
